\documentclass[a4paper,parskip=half, twoside, ngerman, 12pt,BCOR = 11 mm, DIV15, bibliography=totocnumbered]{scrbook}
\usepackage{babel}
\usepackage[utf8]{inputenc}
\usepackage[T1]{fontenc}
\usepackage{calc} %Berechnungen innerhalb von TIKZ/PGFPLOTS
\usepackage{lipsum} %lorem ipsum Text

%Mathematik
\usepackage{amsmath, mathrsfs} 

%Referenzierung
\usepackage{hyperref}

%Grafik
\usepackage{pgfplots}
\pgfplotsset{compat=1.9} %Kompatibilitätsmodus für pgfplots
\usepackage{graphicx}
\graphicspath{{./plot/}}

% Formatierungen
\linespread{1.5} 	%Zeilenabstand
\usepackage{xspace}	%Abstandspaket für Abkürzungen
\raggedbottom		%bündelt den Text von oben nach unten

%Farbe 
\usepackage{color}

%Literaturverzeichnis

%Neue Kommandos
%Scri+ - wird in der TIKZ-Grafik verwendet
\newcommand{\scri}{$\mathscr{I}^+$}

\newcommand{\zB}{z.\,B.\xspace}
\renewcommand{\dh}{d.\,h.\xspace}

\newcommand\picturescale{0.9}
\newcommand\einzelbild{0.6}
