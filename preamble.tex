\documentclass[a4paper,parskip=half, twoside, ngerman, 12pt,BCOR = 11 mm, DIV15, bibliography=totocnumbered]{scrbook}
\usepackage{babel}
\usepackage[utf8]{inputenc}
\usepackage[T1]{fontenc}
\usepackage{calc}
\usepackage{lipsum}
%------------------------------------------------------------------------------
%Mathematik
\usepackage{lmodern,amsmath,amssymb,amsthm, mathrsfs, tensor}
\numberwithin{equation}{chapter} %Setzt den equation-Zaehler nach jeder Section zurueck
\renewcommand{\theequation}{\arabic{chapter}.\arabic{equation}} %Definiert den Stil
\relpenalty=9999 %Penalty fuer inline-math-mode Zeilenumbruch nach Relationen
\binoppenalty=9999 %Penalty fuer inline-math-mode Zeilenumbruch nach Binaeren Operatoren
%------------------------------------------------------------------------------
%Referenzierung
\usepackage{hyperref}
%------------------------------------------------------------------------------
%Grafik
\usepackage{pgfplots}
\pgfplotsset{compat=1.9}
\usepackage{graphicx}
\graphicspath{{./plot/}}
\usepackage{svg}
%------------------------------------------------------------------------------
% Formatierungen
%1,5 facher Zeilenabstand
\linespread{1.5}
\setlength{\footnotesep}{\baselineskip}
\usepackage{xspace, placeins, verbatim}
\raggedbottom

%------------------------------------------------------------------------------
%Farbe 
\usepackage{color}
%------------------------------------------------------------------------------
%Literaturverzeichnis

%------------------------------------------------------------------------------

%Neue Kommandos
%vierdimensionale physikalische Objekte sind fett + mathcal
\newcommand{\RZ}[1]{\boldsymbol{#1}} %RZ=Raumzeit
%Scri+
\newcommand{\scri}{$\mathscr{I}^+$}
\newcommand{\scrimath}{{\mathscr{I}^+}}
\newcommand{\rh}{r_{\mathrm h}} 
\newcommand{\sh}{\sigma_{\mathrm h}}
\newcommand{\sH}{\sigma_{\mathcal H}}
\newcommand{\dsH}{\Delta\sigma_{\mathcal H}}
\newcommand{\aH}{{\hat{\mathcal H}}}
\newcommand{\Ah}{{A_{\mathrm h}}}
\newcommand{\gG}[5]{\tilde\gamma^{#1#2}\tilde\Gamma^{#3}_{#4#5}}
\newcommand{\e}{\mathrm{e}}

\newcommand{\zB}{z.\,B.\xspace}
\renewcommand{\dh}{d.\,h.\xspace}

\newcommand\picturescale{0.9}
\newcommand\einzelbild{0.6}
